\begin{frame}[fragile]{Remote Workflow}
	\begin{tikzpicture}
        [area/.style={rectangle, draw, minimum height= 4em,minimum width= 7em,rounded corners=0.4em,align=center},
        hidden/.style={rectangle, minimum height= 4em,minimum width= 7em,rounded corners=0.4em,align=center}]
				
		\node[area] (working) {Working\\Directory};
		\node[area] (staging) [right=of working] {Staging\\Area};
		\node[area] (local) [right=of staging] {Local\\Repository};
        \node[area] (remote) [above=of local] {Remote\\Repository};
%        \draw[->] (working.south) -- ++(0,-1) --  node[midway, below]{git add} ++ (3.5,0) -| (staging.south) ;
%        \draw[->] (staging.north) -- ++(0,+1) --  node[midway, above]{git commit -m} ++ (3.5,0) -| (local.north) ;

		\begin{scope}[on background layer]
			\node [draw=gray, dashed,fit=(working) (staging) (local), minimum height = 8em, label=above left:{Dev Environment}] {};
		\end{scope}            
	\end{tikzpicture}
\end{frame}

\begin{frame}[fragile]{Why remote?}
    Working on a remote repository allows to:
    \begin{itemize}[<+->]
        \item Maintain an online backup of your work and the entire history of changes.
        \item Collaborate with multiple users on the same files.
        \item Share your work with a community or publicly.
    \end{itemize}
\end{frame}

\begin{frame}[fragile]{Github}
    empty
\end{frame}

\begin{frame}[fragile]{git remote add origin}
    \begin{itemize}
        \item \code{git remote add origin <url_online_repository>} will link your local repository with a remote repo.
        \item \code{git push -u origin master} will syncronize the repositories by \alert{pushing} the local changes, and set the default remote stream.
    \end{itemize}
    \begin{block}{}
        You will be promted to enter the username and password of your remote repository.
        If using 2FA on github, instead of the password you should enter a \href{https://docs.github.com/en/github/authenticating-to-github/creating-a-personal-access-token}{Personal Access Token}.
    \end{block}
\end{frame}

\begin{frame}[fragile]{git push}
    empty
\end{frame}

\begin{frame}[fragile]{git pull}
    empty
\end{frame}

\begin{frame}[fragile]{git clone}
    empty
\end{frame}

\begin{frame}[fragile]{git fetch}
    empty
\end{frame}
