\begin{frame}[fragile]{What is Git}
			
	Git is a tool that protects yourself and others from yourself and others.
		
	\uncover <2->{Git can:}
	\begin{itemize}
		\item <2-> Keep record of all the (tracked) files in your directory.
		\item <3-> Maintain an history of all the changes.
		\item <4-> Revert changes to fix mistakes.
		\item <5-> Allow for concurrent work, helping prevent conflicts.
	\end{itemize}
		
	\uncover <6->{Also known as \alert{Version Control}}
			
\end{frame}

\begin{frame}[fragile]{git init}
	\begin{itemize}[<+->]
		\item  \code{git init} initializes the working directory.
		      		      
		\item It tells git to start managing a \alert{repository} for your workspace.
		      		       
		\item The repository (or 'repo' for short) is a folder in a working directory in which Git tracks all changes made to files and builds a history of those changes.
		      		       
	\end{itemize}
\end{frame}

\begin{frame}[fragile]{git status}
    \code{git status} describes the current state of the repository.

	\begin{shellblock}<1->{
\uncover<1->{~/workspace\$ git init\inserthitenter}\uncover<2->{
Initialized empty Git repository in ~/workspace/.git/
~/workspace\$}\uncover<3->{ git status\inserthitenter}\uncover<4->{
On branch master

No commits yet

nothing to commit (create/copy files and use "git add" to track)
~/workspace\$}\uncover<5->{ rm -r .git\inserthitenter
~/workspace\$}\uncover<6->{ git status\inserthitenter
}\uncover<7->{fatal: not a git repository (or any of the parent directories): .git
    }}\end{shellblock}
		      		       
\end{frame}

\begin{frame}[fragile]{Local Workflow}
	\begin{tikzpicture}
        [area/.style={rectangle, draw, minimum height= 4em,minimum width= 7em,rounded corners=0.4em,align=center},
        hidden/.style={rectangle, minimum height= 4em,minimum width= 7em,rounded corners=0.4em,align=center}]
				
		\node[area] (working) {Working\\Directory};
		\node[hidden] (staging) [right=of working] {};%{Staging\\Area};
		\node[area] (local) [right=of staging] {Local\\Repository};
				
		\begin{scope}[on background layer]
			\node [draw=gray, dashed,fit=(working) (staging) (local), minimum height = 8em, label=above left:{Dev Environment}] {};
		\end{scope}            
	\end{tikzpicture}
\end{frame}

\begin{frame}[fragile]{Untracked files}
	Files in the working directory are not automatically added to the repository.
	\begin{shellblock}<1->{
\uncover<1->{~/workspace\$ touch main.cpp && ls\inserthitenter}\uncover<2->{
main.cpp
~/workspace\$}\uncover<3->{ git status\inserthitenter}\uncover<4->{
On branch master

No commits yet

Untracked files:
    (use "git add <file>..." to include in what will be committed)
        main.cpp

nothing added to commit but untracked files present (use "git add" to 
track)}}\end{shellblock}
		      		       
\end{frame}

\begin{frame}[fragile]{Local Workflow}
	\begin{tikzpicture}
        [area/.style={rectangle, draw, minimum height= 4em,minimum width= 7em,rounded corners=0.4em,align=center},
        hidden/.style={rectangle, minimum height= 4em,minimum width= 7em,rounded corners=0.4em,align=center}]
				
		\node[area, label={[font=\fontsize{0.7em}{0.7em}\selectfont, align=left]below: -tracked files\\-untracked files}] (working) {Working\\Directory};
		\node[area, label={[font=\fontsize{0.7em}{0.7em}\selectfont, align=left]below: -tracked files}] (staging) [right=of working] {Staging\\Area};
		\node[area] (local) [right=of staging] {Local\\Repository};
				
		\begin{scope}[on background layer]
			\node [draw=gray, dashed,fit=(working) (staging) (local), minimum height = 8em, label=above left:{Dev Environment}] {};
		\end{scope}            
	\end{tikzpicture}
\end{frame}

\begin{frame}[fragile]{git add}
    \begin{itemize} 
        \item \code{git add <filename>} asks Git to track changes to a file in the repository. 
    \end{itemize}
    \begin{shellblock}<1->{
\uncover<1->{~/workspace\$ git add main.cpp\inserthitenter}\uncover<2->{
~/workspace\$ git status\inserthitenter}\uncover<3->{
On branch master

No commits yet

Changes to be committed:
    (use "git rm --cached <file>..." to unstage)
        new file:   main.cpp}}\end{shellblock}    		       
\end{frame}

\begin{frame}[fragile]{git add \insertcontinuationtext}
    \begin{itemize} 
        \item You can add multiple files to the staging area with \\\code{git add <file1> <file2> <\dots>}
        \item There is \alert{never} a good reason to use \code{git add *}, \code{git add .}, \code{git add -A} or \code{git add -u}
    \end{itemize}      		       
\end{frame}

\begin{frame}[fragile]{Local Workflow}
	\begin{tikzpicture}
        [area/.style={rectangle, draw, minimum height= 4em,minimum width= 7em,rounded corners=0.4em,align=center},
        hidden/.style={rectangle, minimum height= 4em,minimum width= 7em,rounded corners=0.4em,align=center}]
				
		\node[area] (working) {Working\\Directory};
		\node[area] (staging) [right=of working] {Staging\\Area};
		\node[area] (local) [right=of staging] {Local\\Repository};
        
        \draw[->] (working.south) -- ++(0,-1) --  node[midway, below]{git add} ++ (3.5,0) -| (staging.south) ;
        %\draw[->] (staging.north) -- ++(0,+1) --  node[midway, above]{git commit} ++ (3.5,0) -| (local.north) ;

		\begin{scope}[on background layer]
			\node [draw=gray, dashed,fit=(working) (staging) (local), minimum height = 8em, label=above left:{Dev Environment}] {};
		\end{scope}            
	\end{tikzpicture}
\end{frame}

\begin{frame}[fragile]{git commit -m}
    \begin{itemize} 
        \item Files in the staging area are tracked by Git, but their changes have not been saved on the repository.
        \item With \code{git commit} a collection of changes in the staging area is checked to the local repository and saved to the repo history.
        \item Each commit should be labelled with a message that clearly describes the changes made.
        \item The commit message can be spficied between quotation marks:\\\code{   git commit -m "Commit message"}
    \end{itemize}      		       
\end{frame}

\begin{frame}[fragile]{git commit -m \insertcontinuationtext}
    \begin{itemize} 
        \item \code{git commit -m "Commit message"} asks Git to save changes in the local repository's history. 
    \end{itemize}
    \begin{shellblock}<1->{
\uncover<1->{~/workspace\$ git commit -m "Initial commit"\inserthitenter }\uncover<2->{
[master (root-commit) 0367896] Initial commit
 1 file changed, 0 insertions(+), 0 deletions(-)
 create mode 100644 main.cpp
/workspace\$ git status\inserthitenter}\uncover<3->{
On branch master
nothing to commit, working tree clean}}\end{shellblock}    		       
\end{frame}

\begin{frame}[fragile]{Local Workflow}
	\begin{tikzpicture}
        [area/.style={rectangle, draw, minimum height= 4em,minimum width= 7em,rounded corners=0.4em,align=center},
        hidden/.style={rectangle, minimum height= 4em,minimum width= 7em,rounded corners=0.4em,align=center}]
				
		\node[area] (working) {Working\\Directory};
		\node[area] (staging) [right=of working] {Staging\\Area};
		\node[area] (local) [right=of staging] {Local\\Repository};
        
        \draw[->] (working.south) -- ++(0,-1) --  node[midway, below]{git add} ++ (3.5,0) -| (staging.south) ;
        \draw[->] (staging.north) -- ++(0,+1) --  node[midway, above]{git commit -m} ++ (3.5,0) -| (local.north) ;

		\begin{scope}[on background layer]
			\node [draw=gray, dashed,fit=(working) (staging) (local), minimum height = 8em, label=above left:{Dev Environment}] {};
		\end{scope}            
	\end{tikzpicture}
\end{frame}

\begin{frame}[fragile]{git log}
    \begin{itemize} 
        \item \code{git log} show the local repository's history. 
    \end{itemize}
    \begin{shellblock}<1->{
\uncover<1->{/workspace\$ git log\inserthitenter
commit 03678964f710e1ef1e9d6c0704e3f02f014109a0 (HEAD -> master)
Author: Simone Rossi Tisbeni <simone.rossitisbeni@unibo.it>
Date:   Fri Jan 29 14:39:12 2021 +0000

    Initial commit}}\end{shellblock}    		       
\end{frame}

\begin{frame}[fragile]{git diff}
    \begin{itemize} 
        \item \code{git diff} is used to show the changes in the files.
        \item When no argument is passed, it shows the difference between the current state of the repository and the working directory.
        \item When \code{--staged} is passed as argument, it shows the difference between the current state of the repository and the staged changes.
        \item When a commit number is passed as argument, it shows the difference between the current state of the repository and the selected commit.
    \end{itemize}      		       
\end{frame}

\begin{frame}[fragile]{Remote Workflow}
	\begin{tikzpicture}
        [area/.style={rectangle, draw, minimum height= 4em,minimum width= 7em,rounded corners=0.4em,align=center},
        hidden/.style={rectangle, minimum height= 4em,minimum width= 7em,rounded corners=0.4em,align=center}]
				
		\node[area] (working) {Working\\Directory};
		\node[area] (staging) [right=of working] {Staging\\Area};
		\node[area] (local) [right=of staging] {Local\\Repository};
        \node[area] (remote) [above=of local] {Remote\\Repository};
%        \draw[->] (working.south) -- ++(0,-1) --  node[midway, below]{git add} ++ (3.5,0) -| (staging.south) ;
%        \draw[->] (staging.north) -- ++(0,+1) --  node[midway, above]{git commit -m} ++ (3.5,0) -| (local.north) ;

		\begin{scope}[on background layer]
			\node [draw=gray, dashed,fit=(working) (staging) (local), minimum height = 8em, label=above left:{Dev Environment}] {};
		\end{scope}            
	\end{tikzpicture}
\end{frame}