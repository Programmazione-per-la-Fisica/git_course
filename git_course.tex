\documentclass[usepdftitle=false]{beamer}
\usepackage[utf8]{inputenc}
\usepackage[T1]{fontenc}
\usepackage{lmodern}
\usepackage{underscore}
\usepackage{soul} %% for strikethrough
\usepackage{upquote} % for single quotes in verbatim environments 
\usepackage{tikz}
\usetikzlibrary{quotes}
\usetikzlibrary{positioning}
\usetikzlibrary{arrows.meta}
\usetikzlibrary{shapes.geometric}
\usetikzlibrary{shapes.multipart}
\usetikzlibrary{shapes.symbols}
\usetikzlibrary{shapes.arrows}
\usetikzlibrary{shapes.misc}
\usetikzlibrary{shapes.multipart}
\usetikzlibrary{matrix}
\usetikzlibrary{intersections}
\usetikzlibrary{fit}
\usetikzlibrary{backgrounds}
\usepackage{microtype}
\DisableLigatures{encoding = T1, family = tt*}
\usepackage[absolute,overlay]{textpos}
%\usepackage[english]{babel}
%\usepackage{times}
%\usepackage[scaled]{beramono}
\usepackage{textcomp} % to display correctly single quotes inside verbatim

% to create bookmarks in the pdf file
\usepackage{bookmark}
\usepackage{etoolbox}

\newlength{\bytesize}
\newlength{\memwidth}
\newlength{\memheight}

% to be used for visibility of labels
% https://tex.stackexchange.com/questions/99119/beamer-problematic-use-of-visible-and-only-in-combination-with-tikz-to-draw-a
\tikzset{
  invisible/.style={opacity=0},
  visible on/.style={alt=#1{}{invisible}},
  alt/.code args={<#1>#2#3}{%
    \alt<#1>{\pgfkeysalso{#2}}{\pgfkeysalso{#3}} % \pgfkeysalso doesn't change the path
  },
}                               %

\setlength{\TPHorizModule}{1cm}
\setlength{\TPVertModule}{1cm}

%\usepackage{fancyvrb}
%\fvset{fontsize=\footnotesize}
%\RecustomVerbatimEnvironment{semiverbatim}{Verbatim}{}

\newenvironment<>{codeblock}{%
  \begin{actionenv}#1%
    \def\insertblocktitle{}%
    \par%
    \mode<presentation>{%\usebeamerfont{block}%
      \setbeamercolor{local structure}{parent=example text}}%
    \usebeamertemplate{block example begin}%
    \begin{semiverbatim}}
                 {\par%
    \end{semiverbatim}%
    \usebeamertemplate{block example end}%
                 \end{actionenv}}

\newenvironment<>{shellblock}{%
  \begin{actionenv}#1%
    \def\insertblocktitle{}%
    \par%
      \setbeamercolor{block body example}{fg=green,bg=black}%
    \usebeamertemplate{block example begin}%
    \begin{semiverbatim}}
                 {\par%
    \end{semiverbatim}%
    \usebeamertemplate{block example end}%
                 \end{actionenv}}

\mode<presentation>
{
  \usetheme{default}

  %\setbeamercovered{transparent}
  \setbeamerfont{block body example}{size=\scriptsize}
  \setbeamertemplate{blocks}[rounded][shadow=false]
  \setbeamertemplate{itemize item}[circle]
  \setbeamertemplate{itemize subitem}{$\circ$}
  \setbeamertemplate{itemize subsubitem}{$-$}
  \addtobeamertemplate{navigation symbols}{}{%
	  \usebeamerfont{footline}%
	  \usebeamercolor[fg]{footline}%
		\hspace{1em}%
		\insertframenumber
	}
  \usecolortheme{whale} % applies to outer elements
  \definecolor{infnblue}{RGB}{0,45,75}
  \definecolor{red-triadic}{RGB}{75,0,45}
  \definecolor{green-triadic}{RGB}{38,75,0}
  
  \definecolor{brickred}{rgb}{0.8, 0.25, 0.33}
  \definecolor{darkspringgreen}{rgb}{0.09, 0.45, 0.27}
  %% \setbeamercolor*{frametitle}{bg=infnblue,fg=white}
  %% \setbeamercolor*{alerted text}{fg=red-triadic}
   \setbeamercolor*{block body example}{use={frametitle,normal text},fg=normal text.fg,bg=frametitle.bg!5!white}
%  \usecolortheme{rose} % applies only to blocks
}

\renewcommand{\~}{\protect\scalebox{1.2}{\textasciitilde}}
\newcommand{\code}[1]{\texttt{#1}}
\newcommand{\bslash}{\textbackslash}
\newcommand\upquote[1]{\textquotesingle#1\textquotesingle}
\newcommand{\bslashn}{\upquote{\bslash{}n}}
\newcommand{\bslashz}{\upquote{\bslash{}0}}
\newcommand{\opt}[1]{#1\ensuremath{_{\mathit{opt}}}}
\newcommand{\ddd}{\(\cdots\)}
\newcommand{\qnewline}{\textquotesingle\\n\textquotesingle}
\newcommand{\inserthitenter}{\includegraphics[height=1.1ex]{images/newline.png}}

\tikzset{
  memory/.style={rectangle,draw=black!50,minimum width=.9\textwidth,minimum height=.5cm},
  word/.style={rectangle,minimum width=1.cm,minimum height=.5cm,fill=green!20!white,draw=black!50},
  phantom word/.style={rectangle,minimum width=1.cm,minimum height=.5cm},
  word anchor/.style={rectangle,minimum width=0.25cm,minimum height=0.5cm,draw=black,densely dotted}
}

% to set anchors within text to be able reference them from drawings outside the text
%

% semiverbatim with tikz in beamer, https://tex.stackexchange.com/a/165937
\makeatletter
\global\let\tikz@ensure@dollar@catcode=\relax

% to add bookmark to pdf
\apptocmd{\beamer@@frametitle}{
  % add a bookmark that shows up in the PDF TOC at the subsection level
  \only<1>{\bookmark[page=\the\c@page,level=2]{#1}}
}%

\makeatother

\tikzstyle{every picture}+=[remember picture]

\newcommand{\tikzref}[1]{% to define an anchor
  \tikz \coordinate (#1) at (0,0.5ex);%
}


\title{A minimal Git guide}
\hypersetup{pdftitle=A minimal Git guide -- 2020/2021}

\author{S.~Rossi Tisbeni\\\tiny\url{https://www.unibo.it/sitoweb/simone.rossitisbeni/}}

\institute{Programmazione per la Fisica, Università di Bologna}

\date{Anno Accademico 2020/2021}

%\date[Short Occasion] % (optional)
%{Date / Occasion}

%\subject{Talks}
% This is only inserted into the PDF information catalog. Can be left
% out.

% If you have a file called "university-logo-filename.xxx", where xxx
% is a graphic format that can be processed by latex or pdflatex,
% resp., then you can add a logo as follows:

%\pgfdeclareimage[height=0.5cm]{by-sa}{by-sa}
%\logo{\pgfuseimage{by-sa}}

\AtBeginSection[] % Do nothing for \section*
{
  \begin{frame}<beamer>
    \frametitle{Outline}
    \tableofcontents[currentsection]
  \end{frame}
}

% If you wish to uncover everything in a step-wise fashion, uncomment
% the following command:

%\beamerdefaultoverlayspecification{<+->}

\begin{document}


\begin{frame}[plain]
	\begin{tikzpicture}[remember picture,overlay]
		\node (by-sa) at ([yshift=1cm]current page.south) {\includegraphics[height=.5cm]{images/by-sa}};
		\node (git) at ([yshift=.2cm]by-sa.north) {\tiny\url{https://baltig.infn.it/giaco/pf2020}};
		\node (iol) at ([yshift=.2cm]git.north) {\tiny\url{https://virtuale.unibo.it/course/view.php?id=18455}};
	\end{tikzpicture}
	\titlepage
\end{frame}
  
\begin{frame}[fragile]{What is Git}
			
	Git is a tool that protects yourself and others from yourself and others.
		
	\uncover <2->{Git can:}
	\begin{itemize}
		\item <2-> Keep record of all the (tracked) files in your directory.
		\item <3-> Maintain a history of all the changes.
		\item <4-> Revert changes to fix mistakes.
		\item <5-> Allow for concurrent work and help preventing conflicts.
	\end{itemize}
		
	\uncover <6->{Also known as \alert{Version Control}}
			
\end{frame}

\begin{frame}[fragile]{git init}
	\begin{itemize}[<+->]
		\item  \code{git init} initializes the working directory.
		      		      
		\item It tells Git to start managing a \alert{repository} for your workspace.
		      		       
		\item The repository (or 'repo' for short) is a folder in a working directory in which Git tracks all changes made to files and builds a history of those changes.
		      		       
	\end{itemize}
\end{frame}

\begin{frame}[fragile]{git status}
    \code{git status} describes the current state of the repository.

	\begin{shellblock}<1->{
\uncover<1->{~/workspace\$ git init\inserthitenter}\uncover<2->{
Initialized empty Git repository in ~/workspace/.git/
~/workspace\$}\uncover<3->{ git status\inserthitenter}\uncover<4->{
On branch main

No commits yet

nothing to commit (create/copy files and use "git add" to track)
~/workspace\$}\uncover<5->{ rm -r .git\inserthitenter
~/workspace\$}\uncover<6->{ git status\inserthitenter
}\uncover<7->{fatal: not a git repository (or any of the parent directories): .git
    }}\end{shellblock}
		      		       
\end{frame}

\begin{frame}[fragile]{Local Workflow}
	\begin{tikzpicture}
        [area/.style={rectangle, draw, minimum height= 4em,minimum width= 7em,rounded corners=0.4em,align=center},
        hidden/.style={rectangle, minimum height= 4em,minimum width= 7em,rounded corners=0.4em,align=center}]
				
		\node[area] (working) {Working\\Directory};
		\node[hidden] (staging) [right=of working] {};%{Staging\\Area};
		\node[area] (local) [right=of staging] {Local\\Repository};
				
		\begin{scope}[on background layer]
			\node [draw=gray, dashed,fit=(working) (staging) (local), minimum height = 8em, label=above left:{Dev Environment}] {};
		\end{scope}            
	\end{tikzpicture}
\end{frame}

\begin{frame}[fragile]{Untracked files}
	Files in the working directory are not automatically added to the repository.
	\begin{shellblock}<1->{
\uncover<1->{~/workspace\$ touch main.cpp && ls\inserthitenter}\uncover<2->{
main.cpp
~/workspace\$}\uncover<3->{ git status\inserthitenter}\uncover<4->{
On branch main

No commits yet

Untracked files:
    (use "git add <file>..." to include in what will be committed)
        main.cpp

nothing added to commit but untracked files present (use "git add" to 
track)}}\end{shellblock}
		      		       
\end{frame}

\begin{frame}[fragile]{Local Workflow}
	\begin{tikzpicture}
        [area/.style={rectangle, draw, minimum height= 4em,minimum width= 7em,rounded corners=0.4em,align=center},
        hidden/.style={rectangle, minimum height= 4em,minimum width= 7em,rounded corners=0.4em,align=center}]
				
		\node[area, label={[font=\fontsize{0.7em}{0.7em}\selectfont, align=left]below: -tracked files\\-untracked files}] (working) {Working\\Directory};
		\node[area, label={[font=\fontsize{0.7em}{0.7em}\selectfont, align=left]below: -tracked files}] (staging) [right=of working] {Staging\\Area};
		\node[area] (local) [right=of staging] {Local\\Repository};
				
		\begin{scope}[on background layer]
			\node [draw=gray, dashed,fit=(working) (staging) (local), minimum height = 8em, label=above left:{Dev Environment}] {};
		\end{scope}            
	\end{tikzpicture}
\end{frame}

\begin{frame}[fragile]{git add}
    \begin{itemize} 
        \item \code{git add <filename>} asks Git to track changes to a file in the repository. 
    \end{itemize}
    \begin{shellblock}<1->{
\uncover<1->{~/workspace\$ git add main.cpp\inserthitenter}\uncover<2->{
~/workspace\$ git status\inserthitenter}\uncover<3->{
On branch main

No commits yet

Changes to be committed:
    (use "git rm --cached <file>..." to unstage)
        new file:   main.cpp}}\end{shellblock}    		       
\end{frame}

\begin{frame}[fragile]{git add \insertcontinuationtext}
    \begin{itemize}[<+->] 
        \item You can add multiple files to the staging area with \\\code{git add <file1> <file2> <\dots>}
        \item There is \alert{never} a good reason to use \code{git add *}, \code{git add .} or \code{git add -A}.
    \end{itemize}    	
	\uncover<3->{Git is used to track changes in text files. 
	
	The functionality of Git is reduced for binary files.}	       
\end{frame}

\begin{frame}[fragile]{Local Workflow}
	\begin{tikzpicture}
        [area/.style={rectangle, draw, minimum height= 4em,minimum width= 7em,rounded corners=0.4em,align=center},
        hidden/.style={rectangle, minimum height= 4em,minimum width= 7em,rounded corners=0.4em,align=center}]
				
		\node[area] (working) {Working\\Directory};
		\node[area] (staging) [right=of working] {Staging\\Area};
		\node[area] (local) [right=of staging] {Local\\Repository};
        
        \draw[->] (working.south) -- ++(0,-1) --  node[midway, below]{git add} ++ (3.5,0) -| (staging.south) ;
        %\draw[->] (staging.north) -- ++(0,+1) --  node[midway, above]{git commit} ++ (3.5,0) -| (local.north) ;

		\begin{scope}[on background layer]
			\node [draw=gray, dashed,fit=(working) (staging) (local), minimum height = 8em, label=above left:{Dev Environment}] {};
		\end{scope}            
	\end{tikzpicture}
\end{frame}

\begin{frame}[fragile]{git commit -m}
    \begin{itemize} 
        \item Files in the staging area are tracked by Git, but their changes have not been saved into the repository.
        \item With \code{git commit} a collection of changes in the staging area is checked into the local repository and inserted into the repo history.
        \item Each commit should be labelled with a message that clearly describes the changes made.
        \item The commit message can be specified between quotation marks:\\\code{   git commit -m "Commit message"}
    \end{itemize}      		       
\end{frame}

\begin{frame}[fragile]{git commit -m \insertcontinuationtext}
    \begin{itemize} 
        \item \code{git commit -m "Commit message"} asks Git to save changes into the local repository history. 
    \end{itemize}
    \begin{shellblock}<1->{
\uncover<1->{~/workspace\$ git commit -m "Initial commit"\inserthitenter }\uncover<2->{
[main (root-commit) 0367896] Initial commit
 1 file changed, 0 insertions(+), 0 deletions(-)
 create mode 100644 main.cpp
/workspace\$ git status\inserthitenter}\uncover<3->{
On branch main
nothing to commit, working tree clean}}\end{shellblock}
\begin{block}<3->{}
	If this is your first time working with Git, you will be prompted to enter your email and name, with which to sign your commits.
	
	Follow the instructions, then redo the commit.
\end{block}    		       
\end{frame}

\begin{frame}[fragile]{Local Workflow}
	\begin{tikzpicture}
        [area/.style={rectangle, draw, minimum height= 4em,minimum width= 7em,rounded corners=0.4em,align=center},
        hidden/.style={rectangle, minimum height= 4em,minimum width= 7em,rounded corners=0.4em,align=center}]
				
		\node[area] (working) {Working\\Directory};
		\node[area] (staging) [right=of working] {Staging\\Area};
		\node[area] (local) [right=of staging] {Local\\Repository};
        
        \draw[->] (working.south) -- ++(0,-1) --  node[midway, below]{git add} ++ (3.5,0) -| (staging.south) ;
        \draw[->] (staging.north) -- ++(0,+1) --  node[midway, above]{git commit} ++ (3.5,0) -| (local.north) ;

		\begin{scope}[on background layer]
			\node [draw=gray, dashed,fit=(working) (staging) (local), minimum height = 8em, label=above left:{Dev Environment}] {};
		\end{scope}            
	\end{tikzpicture}
\end{frame}

\begin{frame}[fragile]{git log}
    \begin{itemize} 
        \item \code{git log} shows the local repository history. 
    \end{itemize}
    \begin{shellblock}<1->{
\uncover<1->{/workspace\$ git log\inserthitenter
commit 03678964f710e1ef1e9d6c0704e3f02f014109a0 (HEAD -> main)
Author: Simone Rossi Tisbeni <simone.rossitisbeni@unibo.it>
Date:   Fri Jan 29 14:39:12 2021 +0000

    Initial commit}}\end{shellblock}
	\uncover<2->{
		\begin{itemize}
			\item Each commit has its own unique identifier.
		\end{itemize}
	}    		       
\end{frame}

\begin{frame}[fragile]{git diff}
    \begin{itemize} 
        \item \code{git diff} is used to show the changes in the files.
        \item When no argument is passed, it shows any uncommitted changes since the last commit.
        \item When \code{--staged} is passed as argument, it shows the difference between the last commit and the staged changes.
        \item When a commit id is passed as argument, it shows the difference between the last commit and the selected commit.
    \end{itemize}      		       
\end{frame}

\begin{frame}[fragile]{Branching}
\begin{itemize}[<+->]
	\item   A git repositry keeps track of the history of changes.
    \item   It can also keep track of multiple lines of development.
    \item   These are known as \alert{branches}
\end{itemize}
\end{frame}

\begin{frame}[fragile]{git branch}
    \begin{itemize}[<+->]
        \item \code{git branch} is the command used to manipulate different development lines.
        \item When launched with no arguments, lists the branches available in the repository
    \end{itemize}
    \begin{shellblock}<3->
~/workspace\$ git branch\inserthitenter
{\color{white}*} master
\end{shellblock}
    \begin{itemize}[<4->]
        \item The current branch is highlighted in green and prefixed with \code{*} (asterisk)
    \end{itemize}
\end{frame}

\begin{frame}[fragile]{History Graph}
    \uncover<1>{
        \begin{tikzpicture}[remember picture,overlay,
            nodo/.style={circle, draw, fill},
            linea/.style={line width=0.3em},
            nome/.style={rotate=45}]
            \node[nodo] (comm2) at (current page.center) {};
            \node[nodo] (comm1) [left=of comm2] {};
            \node[nodo] (init) [left=of comm1]{};
            \node[nodo] (comm3) [right=of comm2] {};
            \node[nodo] (head) [right=of comm3] {};
            \draw[linea] (init) -- (comm1);
            \draw[linea] (comm1) -- (comm2);
            \draw[linea] (comm2) -- (comm3);
            \draw[linea] (comm3) -- (head);
            \node[nome] at (init.south west) [anchor=east] {Initial commit};
            \node[nome] at (comm1.south west) [anchor=east] {Commit \#2};
            \node[minimum height=2em] at (comm2.south east) [anchor=north] {\dots};
            \node[nome] at (head.south west) [anchor=east] {HEAD};
        \end{tikzpicture}
    }
    \uncover<2>{
        \begin{tikzpicture}[remember picture,overlay,
            nodo/.style={circle, draw, fill, brickred},
            linea/.style={line width=0.3em, brickred},
            nome/.style={draw}]
            \node[nodo] (comm2) at (current page.center) {};
            \node[nodo] (comm1) [left=of comm2] {};
            \node[nodo] (init) [left=of comm1]{};
            \node[nodo] (comm3) [right=of comm2] {};
            \node[nodo] (head) [right=of comm3] {};
            \draw[linea] (init) -- (comm1);
            \draw[linea] (comm1) -- (comm2);
            \draw[linea] (comm2) -- (comm3);
            \draw[linea] (comm3) -- (head);
            \node[nome, red] (master) [above=of head] {master};
            \draw[->, shorten >=5pt] (master) -- (head);
        \end{tikzpicture}
    }
\end{frame}

\begin{frame}[fragile]{Main branch}
    \begin{itemize}[<+->]
        \item The default branch created at the initialization of the repository is usually named \alert{master}.
        \item The deafult is slowly moving towards a more inclusive denomination.
        \item Github already started using (Oct 2020) the name \alert{main}.
    \end{itemize}
    \begin{shellblock}<4->
~/workspace\$ git branch -m master main\inserthitenter
~/workspace\$ git branch\inserthitenter
{\color{white}*} main
\end{shellblock}
\end{frame}

\begin{frame}[fragile]{git branch \insertcontinuationtext}
    \begin{itemize}
        \item You can create a new branch using \code{git branch <branch_name>}
    \end{itemize}
    \begin{shellblock}<2->
~/workspace\$ git branch new_branch\inserthitenter
~/workspace\$ git branch\inserthitenter
{\color{white}*} main
{\color{white}  new_branch}    
    \end{shellblock}

    \begin{itemize}[<3->]
        \item \code{git log} shows the position of the branches on the commit history
    \end{itemize}
\end{frame}

\begin{frame}[fragile]{History Graph}
    \uncover<1>{
        \begin{tikzpicture}[remember picture,overlay,
            nodo/.style={circle, draw, fill, brickred},
            linea/.style={line width=0.3em, brickred},
            nome/.style={draw}]
            \node[nodo] (comm2) at (current page.center) {};
            \node[nodo] (comm1) [left=of comm2] {};
            \node[nodo] (init) [left=of comm1]{};
            \node[nodo] (comm3) [right=of comm2] {};
            \node[nodo] (head) [right=of comm3] {};
            \draw[linea] (init) -- (comm1);
            \draw[linea] (comm1) -- (comm2);
            \draw[linea] (comm2) -- (comm3);
            \draw[linea] (comm3) -- (head);
            \node[nome, brickred] (main) [above=of head] {main};
            \draw[->, shorten >=5pt] (main) -- (head);
            \node[nome, darkspringgreen] (branch) [below=of head] {new_branch};
            \draw[->, shorten >=5pt] (branch) -- (head);
        \end{tikzpicture}
    }
\end{frame}

\begin{frame}[fragile]{git checkout }
    \begin{itemize}
        \item The commit histories are independent on different branches
    \end{itemize}
\end{frame}

\begin{frame}[fragile]{git checkout}
    \uncover<1>{
        \begin{tikzpicture}[remember picture,overlay,
            nodo/.style={circle, draw, fill, brickred},
            linea/.style={line width=0.3em, brickred},
            nome/.style={draw}]
            \node[nodo] (comm3) at (current page.center) {};
            \node[nodo] (comm2) [left=of comm3] {};
            \node[nodo] (comm1) [left=of comm2] {};
            \node[nodo] (init) [left=of comm1]{};
            \node[nodo] (head) [right=of comm3] {};
            \draw[linea] (init) -- (comm1);
            \draw[linea] (comm1) -- (comm2);
            \draw[linea] (comm2) -- (comm3);
            \draw[linea] (comm3) -- (head);
            \node[nome, brickred] (main) [above=of head] {main};
            \draw[->, shorten >=5pt] (main) -- (head);
            \node[nome, darkspringgreen] (branch) [below=of head] {new_branch};
            \draw[->, shorten >=5pt] (branch) -- (head);
        \end{tikzpicture}
    }
    \uncover<2>{
        \begin{tikzpicture}[remember picture,overlay,
            nodo/.style={circle, draw, fill, brickred},
            linea/.style={line width=0.3em, brickred},
            nome/.style={draw}]
            \node[nodo] (comm3) at (current page.center) {};
            \node[nodo] (comm2) [left=of comm3] {};
            \node[nodo] (comm1) [left=of comm2] {};
            \node[nodo] (init) [left=of comm1]{};
            \node[nodo] (head) [right=of comm3] {};
            \draw[linea] (init) -- (comm1);
            \draw[linea] (comm1) -- (comm2);
            \draw[linea] (comm2) -- (comm3);
            \draw[linea] (comm3) -- (head);
            \node[nodo] (newcom) [right=of head] {};
            \draw[linea] (head) -- (newcom);
            \node[nome, brickred] (main) [above=of newcom] {main};
            \draw[->, shorten >=5pt] (main) -- (newcom);
            \node[nome, darkspringgreen] (branch) [below=of head] {new_branch};
            \draw[->, shorten >=5pt] (branch) -- (head);
        \end{tikzpicture}
    }
\end{frame}

\begin{frame}[fragile]{git checkout }
    \begin{itemize}
        \item The commit histories are independent on different branches
        \item Any new commit will be available only on the active branch.
        \item <2-> To switch branch, you use \code{git checkout <branch_name>}
    \end{itemize}
\end{frame}

\begin{frame}[fragile]{git checkout}
    \uncover<1>{
        \begin{tikzpicture}[remember picture,overlay,
            nodo/.style={circle, draw, fill, brickred},
            linea/.style={line width=0.3em, brickred},
            nome/.style={draw}]
            \node[nodo] (comm3) at (current page.center) {};
            \node[nodo] (comm2) [left=of comm3] {};
            \node[nodo] (comm1) [left=of comm2] {};
            \node[nodo] (init) [left=of comm1]{};
            \node[nodo] (head) [right=of comm3] {};
            \draw[linea] (init) -- (comm1);
            \draw[linea] (comm1) -- (comm2);
            \draw[linea] (comm2) -- (comm3);
            \draw[linea] (comm3) -- (head);
            \node[nodo] (newcom) [right=of head] {};
            \draw[linea] (head) -- (newcom);
            \node[nome, brickred] (main) [above=of newcom] {main};
            \draw[->, shorten >=5pt] (main) -- (newcom);
            \node[nome, darkspringgreen] (branch) [below=of head] {new_branch};
            \draw[->, shorten >=5pt] (branch) -- (head);
        \end{tikzpicture}
    }
    \uncover<2>{
        \begin{tikzpicture}[remember picture,overlay,
            nodo/.style={circle, draw, fill, brickred},
            linea/.style={line width=0.3em, brickred},
            nome/.style={draw}]
            \node[nodo] (comm3) at (current page.center) {};
            \node[nodo] (comm2) [left=of comm3] {};
            \node[nodo] (comm1) [left=of comm2] {};
            \node[nodo] (init) [left=of comm1]{};
            \node[nodo] (head) [right=of comm3] {};
            \draw[linea] (init) -- (comm1);
            \draw[linea] (comm1) -- (comm2);
            \draw[linea] (comm2) -- (comm3);
            \draw[linea] (comm3) -- (head);
            \node[nodo] (newcom) [right=of head] {};
            \node (none) [below=of newcom] {};
            \node[nodo, darkspringgreen, yshift=10] (commb) [right=of none] {};
            \draw[linea] (head) -- (newcom);
            \draw[linea, darkspringgreen] (head.east) to[out=0,in=180] (commb.west);
            \node[nome, brickred] (main) [above=of newcom] {main};
            \draw[->, shorten >=5pt] (main) -- (newcom);
            \node[nome, darkspringgreen] (branch) [below=of commb] {new_branch};
            \draw[->, shorten >=5pt] (branch) -- (commb);
        \end{tikzpicture}
    }
\end{frame}

\begin{frame}[fragile]{Merge}
    empty
\end{frame}

\begin{frame}[fragile]{Merge conflicts}
    empty
\end{frame}

\begin{frame}[fragile]{git checkout}
    \uncover<1>{
        \begin{tikzpicture}[remember picture,overlay,
            nodo/.style={circle, draw, fill, brickred},
            linea/.style={line width=0.3em, brickred},
            nome/.style={draw}]
            \node[nodo] (head) at (current page.center) {};
            \node[nodo] (comm2) [left=of head] {};
            \node[nodo] (comm1) [left=of comm2] {};
            \node[nodo] (init) [left=of comm1]{};
            \draw[linea] (init) -- (comm1);
            \draw[linea] (comm1) -- (comm2);
            \draw[linea] (comm2) -- (head);
            \node[nodo] (newcom) [right=of head] {};
            \node (none) [below=of newcom] {};
            \node[nodo, darkspringgreen, yshift=10] (commb) [right=of none] {};
            \draw[linea] (head) -- (newcom);
            \draw[linea, darkspringgreen] (head.east) to[out=0,in=180] (commb.west);
            \node[nome, brickred] (main) [above=of newcom] {main};
            \draw[->, shorten >=5pt] (main) -- (newcom);
            \node[nome, darkspringgreen] (branch) [below=of commb] {new_branch};
            \draw[->, shorten >=5pt] (branch) -- (commb);
        \end{tikzpicture}
    }
    \uncover<2>{
        \begin{tikzpicture}[remember picture,overlay,
            nodo/.style={circle, draw, fill, brickred},
            linea/.style={line width=0.3em, brickred},
            nome/.style={draw}]
            \node[nodo] (head) at (current page.center) {};
            \node[nodo] (comm2) [left=of head] {};
            \node[nodo] (comm1) [left=of comm2] {};
            \node[nodo] (init) [left=of comm1]{};
            \draw[linea] (init) -- (comm1);
            \draw[linea] (comm1) -- (comm2);
            \draw[linea] (comm2) -- (head);
            \node[nodo] (newcom) [right=of head] {};
            \node (none) [below=of newcom] {};
            \node[nodo, darkspringgreen, yshift=10] (commb) [right=of none] {};
            \draw[linea] (head) -- (newcom);
            \draw[linea, darkspringgreen] (head.east) to[out=0,in=180] (commb.west);
            \node[circle, draw=none] (hidden) [right=of newcom] {};
            \node[nodo] (merge) [right=of hidden] {};
            \node[nome, brickred] (main) [above=of merge] {main};
            \draw[linea] (newcom) -- (merge);
            \draw[linea, darkspringgreen] (commb.east) to[out=0,in=180] (merge.west);
            \draw[->, shorten >=5pt] (main) -- (merge);
            \node[nome, darkspringgreen] (branch) [below=of merge] {new_branch};
            \draw[->, shorten >=5pt] (branch) -- (merge);
        \end{tikzpicture}
    }
\end{frame}

\begin{frame}[fragile]{references}
    empty
\end{frame}
% \begin{frame}[fragile]{Remote Workflow}
    \begin{tikzpicture}
        [area/.style={rectangle, draw, minimum height= 4em,minimum width= 7em,rounded corners=0.4em,align=center},
            hidden/.style={rectangle, minimum height= 4em,minimum width= 7em,rounded corners=0.4em,align=center}]

        \node[area] (working) {Working\\Directory};
        \node[area] (staging) [right=of working] {Staging\\Area};
        \node[area] (local) [right=of staging] {Local\\Repository};
        \node[area] (remote) [above=of local] {Remote\\Repository};
        %        \draw[->] (working.south) -- ++(0,-1) --  node[midway, below]{git add} ++ (3.5,0) -| (staging.south) ;
        %        \draw[->] (staging.north) -- ++(0,+1) --  node[midway, above]{git commit -m} ++ (3.5,0) -| (local.north) ;

        \begin{scope}[on background layer]
            \node [draw=gray, dashed,fit=(working) (staging) (local), minimum height = 8em, label=above left:{Dev Environment}] {};
        \end{scope}
    \end{tikzpicture}
\end{frame}

\begin{frame}[fragile]{Why remote?}
    Working on a remote repository allows to:
    \begin{itemize}[<+->]
        \item Maintain an online backup of your work and the entire history of changes.
        \item Collaborate with multiple users on the same files.
        \item Share your work with a community or publicly.
    \end{itemize}
\end{frame}

\begin{frame}[fragile]{Github}
    \begin{itemize}[<+->] 
        \item Github is a provider for hosting repositories online for free.
        \item It allows secure access to private and public repositories
        \item To host your repository on Github, first create an account at \url{https://github.com/join}
        \item Then create an empty repository at \url{https://github.com/new}
        \item Don't select any of the initialization options and hit \alert{Create repository}
        \item In the next tab copy the \alert{SSH} link to you repository
    \end{itemize}
\end{frame}

\begin{frame}[fragile]{SSH}
\begin{itemize}[<+->] 
    \item An SSH key is a crypted text that allows for a secure identification of a device.
    \item It is comprised of a pair of files, a public key and a private key.
    \item The private key should never be shared and should always reside on your machine.
    \item The public key is shared with the client that would like to identify your device.
\end{itemize}
\end{frame}

\begin{frame}[fragile]{SSH\insertcontinuationtext}
    To create a SSH key pair run the following command on the bash.
    \begin{shellblock}
~/workspace\$ ssh-keygen -t rsa
Generating public/private rsa key pair.
Enter file in which to save the key (/home/demo/.ssh/id_rsa): 
Enter passphrase (empty for no passphrase): 
Enter same passphrase again: 
Your identification has been saved in /home/demo/.ssh/id_rsa.
Your public key has been saved in /home/demo/.ssh/id_rsa.pub.
The key fingerprint is:
4a:dd:0a:c6:35:4e:3f:ed:27:38:8c:74:44:4d:93:67 demo@a
The key's randomart image is:
+--[ RSA 2048]----+
|          .oo.   |
|         .  o.E  |
|        + .  o   |
...
    \end{shellblock}
Make note of the path to the \code{.pub} file.
\end{frame}

\begin{frame}[fragile]{Github \insertcontinuationtext}
    \begin{itemize}[<+->] 
        \item On Github go to \alert{\href{https://github.com/settings/keys}{SSH and GPG keys}} settings page
        \item Create a New SSH key
        \item Copy the content of the .pub file containing your public key and Add it in the form.
    \end{itemize}
\end{frame}

\begin{frame}[fragile]{git remote add origin}
    \begin{itemize}[<+->] 
        \item \code{git remote add origin <url_online_repository>} will link your local repository with a remote repo.
        \item \code{git push -u origin main} will syncronize the repositories by \alert{pushing} the local changes, and set the default remote stream.
        \item You will be promted to verify the fingerprint and accept the connection.
        \item If you prefer using HTTPS and 2FA on github, instead of SSH keys visit \href{https://docs.github.com/en/github/getting-started-with-github/about-remote-repositories#cloning-with-https-urls}{the official documentation}.
    \end{itemize}
\end{frame}

\begin{frame}[fragile]{git push}
    \begin{itemize}[<+->] 
        \item \code{git push} is the command used to update the remote repository.
        \item When there are one or several commit on the local repository ready to be shared, they have to be pushed to the remote repository.
        \item \code{git push <remote> <branch>} allows to push the content of a local branch to a matching remote branch (i.e \code{git push origin main})
    \end{itemize}
\end{frame}

\begin{frame}[fragile]{git push \insertcontinuationtext}
    \begin{tikzpicture}
        [area/.style={rectangle, draw, minimum height= 4em,minimum width= 7em,rounded corners=0.4em,align=center},
            hidden/.style={rectangle, minimum height= 4em,minimum width= 7em,rounded corners=0.4em,align=center}]

        \node[area] (working) {Working\\Directory};
        \node[area] (staging) [right=of working] {Staging\\Area};
        \node[area] (local) [right=of staging] {Local\\Repository};
        \node[area] (remote) [above=of local] {Remote\\Repository};
        %        \draw[->] (working.south) -- ++(0,-1) --  node[midway, below]{git add} ++ (3.5,0) -| (staging.south) ;
        \draw[->] (local.north) -- (remote.south) node[midway,right]{git push} ;

        \begin{scope}[on background layer]
            \node [draw=gray, dashed,fit=(working) (staging) (local), minimum height = 8em, label=above left:{Dev Environment}] {};
        \end{scope}
    \end{tikzpicture}
\end{frame}

\begin{frame}[fragile]{git fetch}
    \begin{itemize}[<+->] 
        \item \code{git push} fails if the local repository is not up to date with the remote one.
        \item \code{git fetch} allows to pull the metadata from the remote repository to check for difference with the local.
        \item \code{git diff <branch> origin/<branch>} will show the difference between a local branch and a remote one, after fetching the metadata.
    \end{itemize}
\end{frame}

\begin{frame}[fragile]{git fetch \insertcontinuationtext}
    \begin{tikzpicture}
        [area/.style={rectangle, draw, minimum height= 4em,minimum width= 7em,rounded corners=0.4em,align=center},
            hidden/.style={rectangle, minimum height= 4em,minimum width= 7em,rounded corners=0.4em,align=center}]

        \node[area] (working) {Working\\Directory};
        \node[area] (staging) [right=of working] {Staging\\Area};
        \node[area] (local) [right=of staging] {Local\\Repository};
        \node[area] (remote) [above=of local] {Remote\\Repository};
        %        \draw[->] (working.south) -- ++(0,-1) --  node[midway, below]{git add} ++ (3.5,0) -| (staging.south) ;
        \draw[->] (remote.south) -- (local.north) node[midway,right]{git fetch} ;

        \begin{scope}[on background layer]
            \node [draw=gray, dashed,fit=(working) (staging) (local), minimum height = 8em, label=above left:{Dev Environment}] {};
        \end{scope}
    \end{tikzpicture}
\end{frame}

\begin{frame}[fragile]{git pull}
    \begin{itemize}[<+->] 
        \item \code{git pull} will fetch the metadata and also download the current state of the remote into your workspace
        \item It can cause Merge conflicts, that need to be fixed before pushing back to the remote
        \item The complete remote workflow will consist of one or more \alert{commits}, \alert{fetching}, \alert{pulling} and \alert{merging conflicts}, and finally \alert{pushing} to the remote
    \end{itemize}
\end{frame}

\begin{frame}[fragile]{git pull \insertcontinuationtext}
    \begin{tikzpicture}
        [area/.style={rectangle, draw, minimum height= 4em,minimum width= 7em,rounded corners=0.4em,align=center},
            hidden/.style={rectangle, minimum height= 4em,minimum width= 7em,rounded corners=0.4em,align=center}]

        \node[area] (working) {Working\\Directory};
        \node[area] (staging) [right=of working] {Staging\\Area};
        \node[area] (local) [right=of staging] {Local\\Repository};
        \node[area] (remote) [above=of local] {Remote\\Repository};
        \draw[->] (remote.south) -- (local.north);
        \draw [->] ($ (local.north) - (2mm,0) $) -- ++(0,0.2) node[above left]{git pull} -| ($ (working.north) + (2mm,0) $);

        \begin{scope}[on background layer]
            \node [draw=gray, dashed,fit=(working) (staging) (local), minimum height = 8em, label=above left:{Dev Environment}] {};
        \end{scope}
    \end{tikzpicture}
\end{frame}

\begin{frame}[fragile]{git clone}
    \begin{itemize}[<+->] 
        \item A remote repository can be shared between multiple users
        \item To download locally the content of a remote repository you can use \code{git clone <remote_url>}
        \item Cloning and pulling is usually permitted on public repositories, but pushing changes is restricted to users added as collaborators on Github
        \item From your repository web page go to \alert{Settings > Manage access > Invite a collaborator} 
    \end{itemize}
\end{frame}



\end{document}
