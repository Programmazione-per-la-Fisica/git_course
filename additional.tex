\begin{frame}{Additional material}
  
\end{frame}

\begin{frame}[fragile]{Main branch}
  \begin{itemize}[<+->]
      \item Historically, the default branch of the repository was named \alert{master}.
      \item The deafult slowly moved towards a more inclusive denomination.
      \item Github started using (Oct 2020) the name \alert{main}.
      \item Newer versions of git already use \alert{main} as the default name.
      \item For older versions, it is possible to change the name of the default branch using:
  \end{itemize}
  \begin{shellblock}<4->
~/workspace\$ git branch -m master main\inserthitenter
~/workspace\$ git branch\inserthitenter
{\color{white}*} main
\end{shellblock}
\end{frame}

\begin{frame}[fragile]{git commit}
  \begin{itemize}[<+->]
      \item When running \code{git commit} with no arguments, Git will open the default text editor.
      \item This will allow you to create a more complex commit message.
  \end{itemize}
  \begin{codeblock}<3->
Summarize changes in around 50 characters or less

More detailed explanatory text, if necessary. Wrap it to about 72
characters or so. The blank line separating the summary from the body
is critical. Further paragraphs come after blank lines.

Explain the problem that this commit is solving. Focus on why you
are making this change as opposed to how.

If you use an issue tracker, put references to them at the bottom,
like this:

Resolves: #123
See also: #456, #789
  \end{codeblock}
\end{frame}

\begin{frame}[fragile]{Default text editor}
  Git will load the commit message on the default text editor.


  \begin{itemize}[<+->]
      \item Vim is the default editor for many Linux distribution and for Git bash on Windows.
      \item This editor does not show a UI or command shortcut.
      \item You can enter \code{INSERT} mode by hitting \code{I} and then editing the text.
      \item Press \code{ESC} to exit \code{INSERT} mode, followed by \code{:wq} to save changes and quit.
  \end{itemize}


  \begin{itemize}[<+->]
      \item The default on WSL is Nano, a more user-friendly alternative.
      \item This editor shows a list of common shortcuts at the bottom of the Terminal.
      \item Directly edit your file, then hit \code{Ctrl+O} to save and \code{Ctrl+X} to quit. 
  \end{itemize}
\end{frame}

\begin{frame}[fragile]{Default text editor \insertcontinuationtext}
  \begin{itemize}[<+->]
      \item If nano is not installed use \code{sudo apt install nano}
      \item You can change the default text editor for Git with \code{git config --global core.editor nano}
      \item You can also have nano be the default text editor for the OS.
      \item Use \code{nano \~/.bashrc} to edit the startup script.
      \item Add the following lines
      \begin{codeblock}
  export VISUAL=nano
  export EDITOR=nano
      \end{codeblock}
      \item Save and quit with \code{Ctrl+O} \code{Ctrl+X}
  \end{itemize}
\end{frame}

\begin{frame}[fragile]{.gitignore}
  \begin{itemize}[<+->]
      \item Git should be used only for non-generated files.
      \item The output of a compilation (for example) should not be tracked by the repository.
      \item Git will automatically ignore anything listed in a file called \code{.gitignore}
      \begin{codeblock}
# Lines starting with # are comments
# Blank lines can be used as separator for readability

# The pattern below match a specific file which will be excluded
exlcuded-file

# The pattern below match a folder and all its content
build/

# The pattern below matches any file with extension .out
*.out
      \end{codeblock}
  \end{itemize}
\end{frame}

