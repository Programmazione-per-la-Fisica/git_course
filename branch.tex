\begin{frame}[fragile]{Branching}
\begin{itemize}[<+->]
	  \item A Git repository keeps track of the history of changes.
    \item It can also keep track of multiple lines of development.
    \item These are known as \alert{branches}.
\end{itemize}
\end{frame}

\begin{frame}[fragile]{git branch}
    \begin{itemize}[<+->]
        \item \code{git branch} is the command used to manipulate different development lines.
        \item When launched with no arguments, it lists the branches available in the repository.
    \end{itemize}
    \begin{shellblock}<3->
~/workspace\$ git branch\inserthitenter
{\color{white}*} main
\end{shellblock}
    \begin{itemize}[<4->]
        \item The current branch is highlighted in green and prefixed with \code{*} (asterisk).
    \end{itemize}
\end{frame}

\begin{frame}[fragile]{History Graph}
    \uncover<1>{
        \begin{tikzpicture}[remember picture,overlay,
            nodo/.style={circle, draw, fill},
            linea/.style={line width=0.3em},
            nome/.style={rotate=45}]
            \node[nodo] (comm2) at (current page.center) {};
            \node[nodo] (comm1) [left=of comm2] {};
            \node[nodo] (init) [left=of comm1]{};
            \node[nodo] (comm3) [right=of comm2] {};
            \node[nodo] (head) [right=of comm3] {};
            \draw[linea] (init) -- (comm1);
            \draw[linea] (comm1) -- (comm2);
            \draw[linea] (comm2) -- (comm3);
            \draw[linea] (comm3) -- (head);
            \node[nome] at (init.south west) [anchor=east] {Initial commit};
            \node[nome] at (comm1.south west) [anchor=east] {Commit \#2};
            \node[minimum height=2em] at (comm2.south east) [anchor=north] {\dots};
            \node[nome] at (head.south west) [anchor=east] {HEAD};
        \end{tikzpicture}
    }
    \uncover<2>{
        \begin{tikzpicture}[remember picture,overlay,
            nodo/.style={circle, draw, fill, brickred},
            linea/.style={line width=0.3em, brickred},
            nome/.style={draw}]
            \node[nodo] (comm2) at (current page.center) {};
            \node[nodo] (comm1) [left=of comm2] {};
            \node[nodo] (init) [left=of comm1]{};
            \node[nodo] (comm3) [right=of comm2] {};
            \node[nodo] (head) [right=of comm3] {};
            \draw[linea] (init) -- (comm1);
            \draw[linea] (comm1) -- (comm2);
            \draw[linea] (comm2) -- (comm3);
            \draw[linea] (comm3) -- (head);
            \node[nome, red] (main) [above=of head] {main};
            \draw[->, shorten >=5pt] (main) -- (head);
        \end{tikzpicture}
    }
\end{frame}

\begin{frame}[fragile]{git branch \insertcontinuationtext}
    \begin{itemize}
        \item   The default branch created at the initialization of the repository is usually named \alert{main}.
        \item   The default branch should host the latest stable version of the repository.
        \item   Additional branches are used to host feature development, bug fixes and beta releases.    
        \item   They are meant to have a very brief life.
        \item   You can create a new branch using \code{git branch <branch_name>}
    \end{itemize}
    \begin{shellblock}<2->
~/workspace\$ git branch new_branch\inserthitenter
~/workspace\$ git branch\inserthitenter
{\color{white}*} main
{\color{white}  new_branch}    
    \end{shellblock}

    \begin{itemize}[<3->]
        \item \code{git log} shows the position of the branches on the commit history.
    \end{itemize}
\end{frame}

\begin{frame}[fragile]{History Graph}
    \uncover<1>{
        \begin{tikzpicture}[remember picture,overlay,
            nodo/.style={circle, draw, fill, brickred},
            linea/.style={line width=0.3em, brickred},
            nome/.style={draw}]
            \node[nodo] (comm2) at (current page.center) {};
            \node[nodo] (comm1) [left=of comm2] {};
            \node[nodo] (init) [left=of comm1]{};
            \node[nodo] (comm3) [right=of comm2] {};
            \node[nodo] (head) [right=of comm3] {};
            \draw[linea] (init) -- (comm1);
            \draw[linea] (comm1) -- (comm2);
            \draw[linea] (comm2) -- (comm3);
            \draw[linea] (comm3) -- (head);
            \node[nome, brickred] (main) [above=of head] {main};
            \draw[->, shorten >=5pt] (main) -- (head);
            \node[nome, darkspringgreen] (branch) [below=of head] {new_branch};
            \draw[->, shorten >=5pt] (branch) -- (head);
        \end{tikzpicture}
    }
\end{frame}

\begin{frame}[fragile]{git checkout }
    \begin{itemize}
        \item The commit histories are independent on different branches.
    \end{itemize}
\end{frame}

\begin{frame}[fragile]{git checkout}
    \uncover<1>{
        \begin{tikzpicture}[remember picture,overlay,
            nodo/.style={circle, draw, fill, brickred},
            linea/.style={line width=0.3em, brickred},
            nome/.style={draw}]
            \node[nodo] (comm3) at (current page.center) {};
            \node[nodo] (comm2) [left=of comm3] {};
            \node[nodo] (comm1) [left=of comm2] {};
            \node[nodo] (init) [left=of comm1]{};
            \node[nodo] (head) [right=of comm3] {};
            \draw[linea] (init) -- (comm1);
            \draw[linea] (comm1) -- (comm2);
            \draw[linea] (comm2) -- (comm3);
            \draw[linea] (comm3) -- (head);
            \node[nome, brickred] (main) [above=of head] {main};
            \draw[->, shorten >=5pt] (main) -- (head);
            \node[nome, darkspringgreen] (branch) [below=of head] {new_branch};
            \draw[->, shorten >=5pt] (branch) -- (head);
        \end{tikzpicture}
    }
    \uncover<2>{
        \begin{tikzpicture}[remember picture,overlay,
            nodo/.style={circle, draw, fill, brickred},
            linea/.style={line width=0.3em, brickred},
            nome/.style={draw}]
            \node[nodo] (comm3) at (current page.center) {};
            \node[nodo] (comm2) [left=of comm3] {};
            \node[nodo] (comm1) [left=of comm2] {};
            \node[nodo] (init) [left=of comm1]{};
            \node[nodo] (head) [right=of comm3] {};
            \draw[linea] (init) -- (comm1);
            \draw[linea] (comm1) -- (comm2);
            \draw[linea] (comm2) -- (comm3);
            \draw[linea] (comm3) -- (head);
            \node[nodo] (newcom) [right=of head] {};
            \draw[linea] (head) -- (newcom);
            \node[nome, brickred] (main) [above=of newcom] {main};
            \draw[->, shorten >=5pt] (main) -- (newcom);
            \node[nome, darkspringgreen] (branch) [below=of head] {new_branch};
            \draw[->, shorten >=5pt] (branch) -- (head);
        \end{tikzpicture}
    }
\end{frame}

\begin{frame}[fragile]{git checkout }
    \begin{itemize}
        \item The commit histories are independent on different branches.
        \item Any new commit will be available only on the active branch.
        \item <2-> To switch branch, you use \code{git checkout <branch_name>}
    \end{itemize}
\end{frame}

\begin{frame}[fragile]{git checkout}
    \uncover<1>{
        \begin{tikzpicture}[remember picture,overlay,
            nodo/.style={circle, draw, fill, brickred},
            linea/.style={line width=0.3em, brickred},
            nome/.style={draw}]
            \node[nodo] (comm3) at (current page.center) {};
            \node[nodo] (comm2) [left=of comm3] {};
            \node[nodo] (comm1) [left=of comm2] {};
            \node[nodo] (init) [left=of comm1]{};
            \node[nodo] (head) [right=of comm3] {};
            \draw[linea] (init) -- (comm1);
            \draw[linea] (comm1) -- (comm2);
            \draw[linea] (comm2) -- (comm3);
            \draw[linea] (comm3) -- (head);
            \node[nodo] (newcom) [right=of head] {};
            \draw[linea] (head) -- (newcom);
            \node[nome, brickred] (main) [above=of newcom] {main};
            \draw[->, shorten >=5pt] (main) -- (newcom);
            \node[nome, darkspringgreen] (branch) [below=of head] {new_branch};
            \draw[->, shorten >=5pt] (branch) -- (head);
        \end{tikzpicture}
    }
    \uncover<2>{
        \begin{tikzpicture}[remember picture,overlay,
            nodo/.style={circle, draw, fill, brickred},
            linea/.style={line width=0.3em, brickred},
            nome/.style={draw}]
            \node[nodo] (comm3) at (current page.center) {};
            \node[nodo] (comm2) [left=of comm3] {};
            \node[nodo] (comm1) [left=of comm2] {};
            \node[nodo] (init) [left=of comm1]{};
            \node[nodo] (head) [right=of comm3] {};
            \draw[linea] (init) -- (comm1);
            \draw[linea] (comm1) -- (comm2);
            \draw[linea] (comm2) -- (comm3);
            \draw[linea] (comm3) -- (head);
            \node[nodo] (newcom) [right=of head] {};
            \node (none) [below=of newcom] {};
            \node[nodo, darkspringgreen, yshift=10] (commb) [right=of none] {};
            \draw[linea] (head) -- (newcom);
            \draw[linea, darkspringgreen] (head.east) to[out=0,in=180] (commb.west);
            \node[nome, brickred] (main) [above=of newcom] {main};
            \draw[->, shorten >=5pt] (main) -- (newcom);
            \node[nome, darkspringgreen] (branch) [below=of commb] {new_branch};
            \draw[->, shorten >=5pt] (branch) -- (commb);
        \end{tikzpicture}
    }
\end{frame}

\begin{frame}[fragile]{Merge}
\begin{itemize}[<+->]
    \item Merging is Git way of putting a forked history back together.
    \item It is often used to merge a feature developed in a secondary branch to the main branch.
    \item Git can automatically merge commits unless there are changes that conflict in both commit sequences.
\end{itemize}
\end{frame}

\begin{frame}[fragile]{Merge \insertcontinuationtext}
    \begin{itemize}[<+->]
        \item To merge two branches togheter, first checkout to the destination branch
        \item Then use \code{git merge <branch_name>}
    \end{itemize}
    \uncover<3->{\begin{shellblock}
~/workspace\$ git checkout main \inserthitenter

~/workspace\$ git merge new_branch \uncover<4->{\inserthitenter}
\uncover<4>{

~/workspace\$}
\uncover<5->{
Auto-merging main.cpp

CONFLICT (content): Merge conflict in main.cpp

Automatic merge failed; fix conflicts and then commit the result.
    }\end{shellblock}}
\end{frame}

\begin{frame}[fragile]{Merge conflicts}
    \begin{itemize}[<+->]
        \item If both the merging branches changed the same lines of the same file, Git won't be able to figure out which version to use.
        \item Running \code{git status} shows which files have conflicts that need to be resolved.
    \end{itemize}
    \begin{shellblock}<3->
~/workspace\$ git status \inserthitenter
On branch main
You have unmerged paths.
  (fix conflicts and run "git commit")
  (use "git merge --abort" to abort the merge)

Unmerged paths:
  (use "git add <file>..." to mark resolution)
        both modified:   main.cpp

no changes added to commit (use "git add" and/or "git commit -a")    
    \end{shellblock}
\end{frame}

\begin{frame}[fragile]{Merge conflicts \insertcontinuationtext}
    \begin{itemize}[<+->]
        \item Git modifies the content of the affected files with visual indicators.
        \item These mark the conflicting section in the receiving branch and the upcoming changes. 
    \end{itemize}
    \begin{codeblock}<3->
    This content is not affected by the conflict
    <<<<<<< main
    this is conflicting text from main
    =======
    this is conflicting text from the new branch
    >>>>>>> new_branch;
    \end{codeblock}
\end{frame}

\begin{frame}[fragile]{Merge conflicts \insertcontinuationtext}
    \begin{itemize}[<+->]
        \item To fix the conflict, simply edit the files with the changes to maintain.
        \item Git doesn't care for the content of the files, it simply requires you to remove the markers.
        \item Run \code{git add} on the files to tell Git they are resolved.
        \item Finally commit the changes with \code{git commit}.
    \end{itemize}
\end{frame}

\begin{frame}[fragile]{History Graph}
    \uncover<1>{
        \begin{tikzpicture}[remember picture,overlay,
            nodo/.style={circle, draw, fill, brickred},
            linea/.style={line width=0.3em, brickred},
            nome/.style={draw}]
            \node[nodo] (head) at (current page.center) {};
            \node[nodo] (comm2) [left=of head] {};
            \node[nodo] (comm1) [left=of comm2] {};
            \node[nodo] (init) [left=of comm1]{};
            \draw[linea] (init) -- (comm1);
            \draw[linea] (comm1) -- (comm2);
            \draw[linea] (comm2) -- (head);
            \node[nodo] (newcom) [right=of head] {};
            \node (none) [below=of newcom] {};
            \node[nodo, darkspringgreen, yshift=10] (commb) [right=of none] {};
            \draw[linea] (head) -- (newcom);
            \draw[linea, darkspringgreen] (head.east) to[out=0,in=180] (commb.west);
            \node[nome, brickred] (main) [above=of newcom] {main};
            \draw[->, shorten >=5pt] (main) -- (newcom);
            \node[nome, darkspringgreen] (branch) [below=of commb] {new_branch};
            \draw[->, shorten >=5pt] (branch) -- (commb);
        \end{tikzpicture}
    }
    \uncover<2>{
        \begin{tikzpicture}[remember picture,overlay,
            nodo/.style={circle, draw, fill, brickred},
            linea/.style={line width=0.3em, brickred},
            nome/.style={draw}]
            \node[nodo] (head) at (current page.center) {};
            \node[nodo] (comm2) [left=of head] {};
            \node[nodo] (comm1) [left=of comm2] {};
            \node[nodo] (init) [left=of comm1]{};
            \draw[linea] (init) -- (comm1);
            \draw[linea] (comm1) -- (comm2);
            \draw[linea] (comm2) -- (head);
            \node[nodo] (newcom) [right=of head] {};
            \node (none) [below=of newcom] {};
            \node[nodo, darkspringgreen, yshift=10] (commb) [right=of none] {};
            \draw[linea] (head) -- (newcom);
            \draw[linea, darkspringgreen] (head.east) to[out=0,in=180] (commb.west);
            \node[circle, draw=none] (hidden) [right=of newcom] {};
            \node[nodo] (merge) [right=of hidden] {};
            \node[nome, brickred] (main) [above=of merge] {main};
            \draw[linea] (newcom) -- (merge);
            \draw[linea, darkspringgreen] (commb.east) to[out=0,in=180] (merge.west);
            \draw[->, shorten >=5pt] (main) -- (merge);
            \node[nome, darkspringgreen] (branch) [below=of merge] {new_branch};
            \draw[->, shorten >=5pt] (branch) -- (merge);
        \end{tikzpicture}
    }
\end{frame}

\begin{frame}[fragile]{git commit}
    \begin{itemize}[<+->]
        \item When running \code{git commit} with no arguments, Git will open the default text editor.
        \item This will allow you to create a more complex commit message.
    \end{itemize}
    \begin{codeblock}<3->
Summarize changes in around 50 characters or less

More detailed explanatory text, if necessary. Wrap it to about 72
characters or so. The blank line separating the summary from the body
is critical. Further paragraphs come after blank lines.

Explain the problem that this commit is solving. Focus on why you
are making this change as opposed to how.

If you use an issue tracker, put references to them at the bottom,
like this:

Resolves: #123
See also: #456, #789
    \end{codeblock}
\end{frame}

\begin{frame}[fragile]{Default text editor}
    Git will load the commit message on the default text editor.


    \begin{itemize}[<+->]
        \item Vim is the default editor for many Linux distribution and for Git bash on Windows.
        \item This editor does not show a UI or command shortcut.
        \item You can enter \code{INSERT} mode by hitting \code{I} and then editing the text.
        \item Press \code{ESC} to exit \code{INSERT} mode, followed by \code{:wq} to save changes and quit.
    \end{itemize}


    \begin{itemize}[<+->]
        \item The default on WSL is Nano, a more user-friendly alternative.
        \item This editor shows a list of common shortcuts at the bottom of the Terminal.
        \item Directly edit your file, then hit \code{Ctrl+O} to save and \code{Ctrl+X} to quit. 
    \end{itemize}
\end{frame}

\begin{frame}[fragile]{Default text editor \insertcontinuationtext}
    \begin{itemize}[<+->]
        \item If nano is not installed use \code{sudo apt install nano}
        \item You can change the default text editor for Git with \code{git config --global core.editor nano}
        \item You can also have nano be the default text editor for the OS.
        \item Use \code{nano \~/.bashrc} to edit the startup script.
        \item Add the following lines
        \begin{codeblock}
    export VISUAL=nano
    export EDITOR=nano
        \end{codeblock}
        \item Save and quit with \code{Ctrl+O} \code{Ctrl+X}
    \end{itemize}
\end{frame}

\begin{frame}[fragile]{.gitignore}
    \begin{itemize}[<+->]
        \item Git should be used only for non-generated files.
        \item The output of a compilation (for example) should not be tracked by the repository.
        \item Git will automatically ignore anything listed in a file called \code{.gitignore}
        \begin{codeblock}
# Lines starting with # are comments
# Blank lines can be used as separator for readability

# The pattern below match a specific file which will be excluded
exlcuded-file

# The pattern below match a folder and all its content
build/

# The pattern below matches any file with extension .out
*.out
        \end{codeblock}
    \end{itemize}
\end{frame}

\begin{frame}[fragile]{references}
    Git tutorials and guides:
    \begin{itemize}
        \item \url{https://www.atlassian.com/git/tutorials}
        \item \url{https://hsf-training.github.io/analysis-essentials/git/README.html}
    \end{itemize}
    Git cheatsheet
    \begin{itemize}
        \item \url{http://ndpsoftware.com/git-cheatsheet.html}
    \end{itemize}
    How to write a commit message
    \begin{itemize}
        \item \url{https://chris.beams.io/posts/git-commit/}
    \end{itemize}
    If you are interested in Vim
    \begin{itemize}
        \item \url{https://www.openvim.com/}
    \end{itemize}
    
\end{frame}
